\documentclass[10pt,-letter paper]{article}
\usepackage[left=1in, right=0.75in, top=1in, bottom=0.75in]{geometry}
\usepackage{graphicx} % Required for inserting images
\usepackage{siunitx}
\usepackage{setspace}
\usepackage{gensymb}
\usepackage{xcolor}
\usepackage{caption}
%\usepackage{subcaption}
\doublespacing
\singlespacing
\usepackage[none]{hyphenat}
\usepackage{amssymb}
\usepackage{relsize}
\usepackage[cmex10]{amsmath}
\usepackage{mathtools}
\usepackage{amsmath}
\usepackage{commath}
\usepackage{amsthm}
\interdisplaylinepenalty=2500
%\savesymbol{iint}
\usepackage{txfonts}
%\restoresymbol{TXF}{iint}
\usepackage{wasysym}
\usepackage{amsthm}
\usepackage{mathrsfs}
\usepackage{txfonts}
\let\vec\mathbf{}
\usepackage{stfloats}
\usepackage{float}
\usepackage{cite}
\usepackage{cases}
\usepackage{subfig}
%\usepackage{xtab}
\usepackage{longtable}
\usepackage{multirow}
%\usepackage{algorithm}
\usepackage{amssymb}
%\usepackage{algpseudocode}
\usepackage{enumitem}
\usepackage{mathtools}
%\usepackage{eenrc}
%\usepackage[framemethod=tikz]{mdframed}
\usepackage{listings}
%\usepackage{listings}
\usepackage[latin1]{inputenc}
%%\usepackage{color}{   
%%\usepackage{lscape}
\usepackage{textcomp}
\usepackage{titling}
\usepackage{hyperref}
%\usepackage{fulbigskip}   
\usepackage{tikz}
\usepackage{graphicx}
\lstset{
  frame=single,
  breaklines=true
}
\let\vec\mathbf{}
\usepackage{circuitikz}
\usepackage{enumitem}
\usepackage{graphicx}
\usepackage{siunitx}
\let\vec\mathbf{}
\usepackage{enumitem}
\usepackage{graphicx}
\usepackage{enumitem}
\usepackage{tfrupee}
\usepackage{amsmath}
\usepackage{amssymb}
\usepackage{mwe} % for blindtext and example-image-a in example
\usepackage{wrapfig}
\graphicspath{{figs/}}
\providecommand{\cbrak}[1]{\ensuremath{\left\{#1\right\}}}
\providecommand{\brak}[1]{\ensuremath{\left(#1\right)}}
\newcommand{\sgn}{\mathop{\mathrm{sgn}}}
\providecommand{\abs}[1]{\left\vert#1\right\vert}
\providecommand{\res}[1]{\Res\displaylimits_{#1}} 
\providecommand{\norm}[1]{\left\lVert#1\right\rVert}
%\providecommand{\norm}[1]{\lVert#1\rVert}
\providecommand{\mtx}[1]{\mathbf{#1}}
\providecommand{\mean}[1]{E\left[ #1 \right]}
\providecommand{\fourier}{\overset{\mathcal{F}}{ \rightleftharpoons}}
%\providecommand{\hilbert}{\overset{\mathcal{H}}{ \rightleftharpoons}}
\providecommand{\system}{\overset{\mathcal{H}}{ \longleftrightarrow}}
	%\newcommand{\solution}[2]{\textbf{Solution:}{#1}}
%\newcommand{\solution}{\noindent \textbf{Solution: }}
\newcommand{\cosec}{\,\text{cosec}\,}
\providecommand{\dec}[2]{\ensuremath{\overset{#1}{\underset{#2}{\gtrless}}}}
\newcommand{\myvec}[1]{\ensuremath{\begin{pmatrix}#1\end{pmatrix}}}
\newcommand{\myaugvec}[2]{\ensuremath{\begin{amatrix}{#1}#2\end{amatrix}}}
\newcommand{\mydet}[1]{\ensuremath{\begin{vmatrix}#1\end{vmatrix}}}
\title{GATE EE-2016-36}
\author{Sameer kendal}
\date{\today}
\begin{document}
\maketitle

\textbf{Question}. The current state $Q_A$ and $Q_B$ of a two $JK$ flip-flop is $00$. Assume that the clock rise-time is much smaller than the delay of $JK$ flip-flop. The next state of the system is 
\begin{figure}[!ht]
\centering
\resizebox{1\textwidth}{!}{%
\begin{circuitikz}
\tikzstyle{every node}=[font=\large]
\draw [](3.75,9.5) to[short] (7.5,9.5);
\draw [](7.5,9.5) to[short] (7.5,13.25);
\draw [](3.75,13.25) to[short] (3.75,9.5);
\draw [](13.75,13.25) to[short] (13.75,9.5);
\draw [](13.75,9.5) to[short] (17.5,9.5);
\draw [](17.5,13.25) to[short] (17.5,9.5);
\draw [](3.75,13.25) to[short] (3.75,14.5);
\draw [](3.75,14.5) to[short] (7.5,14.5);
\draw [](7.5,14.5) to[short] (7.5,13.25);
\draw [](13.75,13.25) to[short] (13.75,14.5);
\draw [](13.75,14.5) to[short] (17.5,14.5);
\draw [](17.5,14.5) to[short] (17.5,13);
\draw [](2.5,13.25) to[short] (3.75,13.25);
\draw [](2.5,10.75) to[short] (3.75,10.75);
\draw [](2.5,10.75) to[short] (2.5,15);
\draw [](12.5,13.25) to[short] (13.75,13.25);
\draw [](12.5,10.75) to[short] (13.75,10.75);
\draw [](12.5,13.25) to[short] (12.5,10.75);
\draw[] (12.5,13.25) to[short] (10,13.25);
\draw [](10,10.75) to[short, -o] (7.5,10.75);
\draw [](20,10.75) to[short, -o] (17.5,10.75);
\draw [](10,13.25) to[short] (10,10.75);
\draw [](17.5,13.25) to[short] (20,13.25);
\draw [](7.5,13.25) to[short] (8.75,13.25);
\draw [](1.25,8.25) to[short] (9.25,8.25);
\draw [](9.25,8.25) to[short] (11.25,8.25);
\draw [](11.25,8.25) to[short] (11.25,12);
\draw [](11.25,12) to[short] (13.75,12);
\draw [](1.75,12) to[short] (1.75,8.25);
\draw [](1.75,12) to[short] (3.75,12);
\node [font=\LARGE] at (2.25,15.75) {$5V$};
\node [font=\large] at (4,13.75) {$J$};
\node [font=\large] at (14,13.75) {$J$};
\node [font=\large] at (14,10.5) {$K$};
\node [font=\large] at (4,10.5) {$K$};
\node [font=\large] at (19.75,13.75) {$Q_B$};
\node [font=\large] at (0.75,8.5) {$CLK$};
\node [font=\large] at (8,13.5) {$Q_A$};
\node [font=\large] at (8.25,11) {$\overline{Q_A}$};
\draw [short] (3.75,12.25) -- (4.25,12);
\draw [short] (4.25,12) -- (3.75,11.75);
\draw [short] (13.75,12.25) -- (14.25,12);
\draw [short] (14.25,12) -- (13.75,11.75);
\end{circuitikz}
}%

\label{fig:my_label}
\end{figure}
\begin{enumerate}[label=(\Alph*)]
\item $00$ \item $01$ \item $11$ \item $10$
\end{enumerate}

\end{document}\documentclass{article}
\usepackage{graphicx} % Required for inserting images

\title{assign}
\author{Sameer kendal}
\date{February 2024}

\begin{document}

\maketitle

\section{Introduction}

\end{document}
